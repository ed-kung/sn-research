\section{Value-for-value incentives and content quality}\label{sec_v4v}

We now consider our second hypothesis: that the types of content better rewarded by zaps will increase in prevalence over time as users learn that these kinds of content are better rewarded.

To test this hypothesis, we first need a method for categorizing posts into different quality types. Although many classification schemes could be considered, we opted for the simplest approach of a binary classification into ``high quality'' and ``low quality'' posts based on transparently observable content characteristics. The simple and transparent classification scheme assists with the interpretability of the results.

We label posts as ``high'' or ``low'' quality based on three easily observable post characteristics: the number of words in the post, whether the post body contains images or links, and whether the post is a ``link post''. On Stacker News, there are two main types of posts: ``discussion posts'' and ``link posts''. There is no difference between the two in terms of costs. The only technical difference between a link post and a discussion post is that a link post contains a URL next to the title of the post. Users of Stacker News generally consider link posts to be of lower quality than discussion posts because link posts usually reference someone else's work and is less likely to contain original content. The most common type of post on Stacker News is a link post with no text in the post body, i.e. a post that points to another URL with no additional commentary. Users \emph{can} add text to the body of a link post. The most common reasons for doing this are to add quotations from the linked URL (to preview the linked content), or to add the user's own commentary regarding the linked URL. When users add their own commentary, link posts tend to be better rewarded than if the user only adds quotations or if the user doesn't add any text to the post body.

In contrast to a link post, a discussion post does not link to a URL. Discussion posts are usually original content from the poster. There is a wide range of discussion posts: commentary, essays, fiction, poetry, book and movie reviews, discussion questions, math puzzles, and more. Discussion posts tend to be better rewarded than link posts because they contain more of the poster's original thoughts. Figure \ref{fig_post_examples} shows an example of a link post without body text and an example of a discussion post. The link post, which links to a news article without providing further context, has earned zero sats after three hours. By contrast, the discussion post, which is a personal reflection of the user's time in South Africa, earned 10,000 sats in two hours.

Besides whether or not a post is a link post, we consider two additional indicators of quality: the number of words in the post body and whether the post body contains images or links. Posts with a larger number of words usually contain more original thought and more interesting content than posts with fewer words, and are thus usually better rewarded. Images in posts are also an indicator for quality as they help the reader visualize what the poster is writing about. Finally, links in the post body are usually citations or references to provide evidence and context to the matter being discussed in the post. 

Table \ref{tab_quality_determinants} shows how the amount of zaps and the amount of comments a post receives in its first 48 hours is related to the features discussed above. The table shows that all else equal, link posts are indeed more poorly rewarded than discussion posts, both in terms of zaps and in terms of comments. Posts that contain more words are also better rewarded than posts that contain fewer words. Images and links in the post body seem to improve a post's quality when the post contains many words, but not when the post contains fewer words. This may be due to some posts being ``meme posts'' (i.e. a post that contains only an image meme---it would still count for words in the post body because the image URL counts as a word) or posts that are functionally like link posts in the sense that they only contain a few links with little context or discussion.

Based on the results in Table \ref{tab_quality_determinants}, we categorize a post as ``high quality'' if it is either:
\begin{itemize}
\item a discussion post with more than 50 words, or
\item a link post with more than 50 words \emph{and} images or links in the post body;
\end{itemize}
and ``low quality'' otherwise. In addition, we apply the same sample selection criteria as we used in section \ref{sec_pay_to_post} with one difference: we no longer exclude posts made by the owner of the territory the post is in. We no longer exclude these posts because territory owners receive zaps just like anyone else and face similar incentives to make high quality posts in their territory as any other user. With these definitions and sample selection criteria applied, we count \gn{NumberHighQuality} high quality posts and \gn{NumberLowQuality} low quality posts over our data period.

Our hypothesis is that the incentive structures on Stacker News encourage the production of high quality posts over time. Figure \ref{fig_quality_over_time} plots the share of high quality posts by week and shows that the share of high quality posts has indeed been increasing over time. The results are consistent with the hypothesis. The incentive structures on Stacker News do seem to be incentivizing more original and long-form content which are rewarded by users in the form of zaps.







