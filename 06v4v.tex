\section{Value-for-value incentives and content quality}\label{sec_v4v}

We now consider our second hypothesis: that the types of content better rewarded by zaps will increase in prevalence over time as users learn that these kinds of content are better rewarded.

To test this hypothesis, we first need a method for categorizing posts into different quality types. Although many classification schemes could be considered, we opted for the simplest approach of a binary classification into ``high quality'' and ``low quality'' posts based on transparently observable content characteristics. The simple and transparent classification scheme assists with the interpretability of the results.

We label posts as ``high'' or ``low'' quality based on three easily observable post characteristics: the number of words in the post, whether the post body contains images or links, and whether the post is a ``link post''. On Stacker News, there are two main types of posts: ``discussion posts'' and ``link posts''. There is no difference between the two in terms of costs. The only technical difference between a link post and a discussion post is that a link post contains a URL next to the title of the post. Users of Stacker News generally consider link posts to be of lower quality than discussion posts because link posts usually reference someone else's work and is less likely to contain original content. The most common type of post on Stacker News is a link post with no text in the post body, i.e. a post that points to another URL with no additional commentary. Users \emph{can} add text to the body of a link post. The most common reasons for doing this are to add quotations from the linked URL (to preview the linked content), or to add the user's own commentary regarding the linked URL. When users add their own commentary, link posts tend to be better rewarded than if the user only adds quotations or if the user doesn't add any text to the post body.

In contrast to a link post, a discussion post does not link to a URL. Discussion posts are usually original content from the poster. There is a wide range of discussion posts: commentary, essays, fiction, poetry, book and movie reviews, discussion questions, math puzzles, and more. Discussion posts tend to be better rewarded than link posts because they contain more of the poster's original thoughts. Figure \ref{fig_post_examples} shows an example of a link post without body text and an example of a discussion post. The link post, which links to a news article without providing further context, has earned zero sats after three hours. By contrast, the discussion post, which is a personal reflection of the user's time in South Africa, earned 10,000 sats in two hours.

Besides whether or not a post is a link post, we consider two additional indicators of quality: the number of words in the post body and whether the post body contains images or links. Posts with a larger number of words usually contain more original thought and more interesting content than posts with fewer words, and are thus usually better rewarded. Images in posts are also an indicator for quality as they help the reader visualize what the poster is writing about. Finally, links in the post body are usually citations or references to provide evidence and context to the matter being discussed in the post. 

Table \ref{tab_quality_determinants} shows how the amount of zaps and the amount of comments a post receives in its first 48 hours is related to the features discussed above. The table shows that all else equal, link posts are indeed more poorly rewarded than discussion posts, both in terms of zaps and in terms of comments. Posts that contain more words are also better rewarded than posts that contain fewer words. Images and links in the post body seem to improve a post's quality when the post contains many words, but not when the post contains fewer words. This may be due to some posts being ``meme posts'' (i.e. a post that contains only an image meme---it would still count for words in the post body because the image URL counts as a word) or posts that are functionally like link posts in the sense that they only contain a few links with little context or discussion.

Based on the results in Table \ref{tab_quality_determinants}, we categorize a post as ``high quality'' if it is either:
\begin{itemize}
\item a discussion post with more than 50 words, or
\item a link post with more than 50 words \emph{and} images or links in the post body;
\end{itemize}
and ``low quality'' otherwise. In addition, we apply the same sample selection criteria as we used in section \ref{sec_pay_to_post} with one difference: we no longer exclude posts made by the owner of the territory the post is in. We no longer exclude these posts because territory owners receive zaps just like anyone else and face similar incentives to make high quality posts in their territory as any other user. With these definitions and sample selection criteria applied, we count \gn{NumberHighQuality} high quality posts and \gn{NumberLowQuality} low quality posts over our data period.

Figure \ref{fig_quality_over_time} plots the share of high quality posts by week and shows that the share of high quality posts have been increasing over time. This result is consistent with our hypothesis that the value-for-value incentives on Stacker News encourage the production of higher quality content over time. 

We now turn to the next part of the hypothesis: that users are more likely to make high quality posts after learning that such posts are better rewarded. If the hypothesis is true, then the data should show that users are more likely to make high quality posts after they have personally been well zapped for prior high quality posts. To test this, we run regressions of the following form:
\begin{align}
h_{ijt} = \beta s_{ijt} + \delta_j + \gamma_t + \epsilon_{jt} \label{eq_v4v}
\end{align}
where $i$ indexes the post, $j$ indexes the user who made the post, and $t$ indexes the week the post was made.  $h_{ijt}$ is a binary indicator for whether or not post $i$ is high quality, $s_{ijt}$ measures the share of the user's zaps on prior posts that came from high quality posts (i.e. $s_{ijt} = 0.5$ indicates that half of the user's prior zaps came from high quality posts), $\delta_j$ is a user fixed effect, and $\gamma_t$ is a week fixed effect. This defines a linear probability model estimating the probability that a user's next post is high quality based on the share of their prior zaps earned from high quality posts. Our hypothesis is that $\beta>0$, indicating that users who earn a greater share of their zaps from high quality posts are more likely to \emph{subsequently} make high quality posts. The user fixed effects, $\delta_j$, control for the user's baseline preferences over post types. (For example, some users have a greater baseline preference of making high quality content than others.) The week fixed effects, $\gamma_t$, control for global time trends in the behavior of users in the aggregate. Thus, the effect of learning on post quality is identified from each user's individual experiences over time.

Table \ref{tab_v4v_learning} reports the results for regression \eqref{eq_v4v}. Column 1 includes only the share of prior zaps coming from high quality posts as a regressor. Column 2 adds the log of the total amount of prior zaps as an additional control. Column 3 adds user fixed effects and column 4 adds both user and week fixed effects. Adding user fixed effects significantly moderates the estimated effect of learning, but does not eliminate it. The effect is still robust and statistically significant. Using column 4 as our preferred specification, our results suggest that a user who has earned 75\% of their prior zaps from high quality posts is \gn{HighQualityEffect} percentage points more likely to make a high quality post as their next post than a user who has earned 25\% of their prior zaps from high quality posts. The magnitude of this effect is fairly significant, since only 






