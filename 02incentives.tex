\section{Incentive structures on Stacker News} \label{sec_incentives}

From its outset, Stacker News was founded with the idea of using Bitcoin micropayments to incentivize quality content and good behavior. Its motto is ``Stacker News is trying to fix online communities with economics.'' There are a number of incentive structures built into the design of Stacker News. 

\paragraph{Pay to post.} First, users must pay to post. The posting cost is denominated in sats and varies by territory.\footnote{One sat (or Satoshi) is equal to 1/100,000,000th of a Bitcoin.} A territory is a sub-forum on Stacker News, usually topic-specific. As of October 2025, the most popular territories are \texttt{bitcoin}, which focuses on discussions about Bitcoin; \texttt{econ}, which focuses on news and discussion about economics; \texttt{Stacker\_Sports}, which focuses on sports discussion; \texttt{AI}, focused on news and discussion about artificial intelligence; and \texttt{Politics\_And\_Law}, focused primarily on politics and current events. 

Territories are created, owned, and operated by individual users on Stacker News. To create a territory, a user simply has to choose a name for the territory, write a description, and pay a monthly cost to Stacker News to maintain the territory. As of October 2025, the monthly cost to operate a territory is 50,000 sats.\footnote{Territory owners can also pay for 12 months at a time for 500,000 sats, or pay for a territory in perpetuity if they pay 3 million sats.} If a territory owner fails to pay the monthly cost, the territory becomes archived. Users are still able to see posts in an archived territory, but they can no longer post in it. Any user (it doesn't have to be the territory founder) can unarchive a territory simply by choosing to pay the monthly cost.

Once a new territory is founded, users can post in that territory by paying a posting fee which is set by the territory owner. The territory owner earns 70\% of the posting fee as revenue, and the other 30\% goes to a daily rewards pool (more on that later). Territory owners are therefore incentivized to choose a posting fee that optimizes between post quantity, post quality, and revenue earned. If the posting fee is too low, the territory may not earn enough revenue to cover costs, and it may also attract low quality posts. On the other hand, if the fee is too high, users may be deterred from posting in the territory. Users are incentivized to post in the territory that best fits the topic of their post so as to maximize engagement and sats earned net of posting fees. Territory owners can additionally attract users to post in their territories by creating engaging content or by adding value to users' posts. For example, the owner of the  \texttt{Stacker\_Sports} territory often runs sports prediction games for users to engage in, and the owner of \texttt{BooksAndArticles} writes a weekly newsletter to highlight the territory's best posts, thus adding visibility and additional zaps for those creators.

\paragraph{Value for value.} Second, users can tip each other directly for content that they like. In the Bitcoin community, systems that allow users to directly tip each other are called ``value-for-value'' systems. On Stacker News, these tips are called ``zaps''. To zap an item, the user clicks a lightning bolt icon which is featured prominently on every post and comment. When the user clicks the lightning bolt, sats are sent from the user to the creator of the item. The default zap amount is 10 sats, but users can customize their own defaults. They can also long-press the lightning bolt icon to zap a custom amount.

Like Reddit upvotes, zaps are used by the platform to surface high quality content---the default view on Stacker News ranks posts based on the trust-weighted value of zaps earned by the post (more on the trust system later). Unlike Reddit upvotes, zaps represent real monetary value transferred from the zapper to the creator of the post. The zapping system incentivizes high quality content in two ways: not only does the system surface higher quality posts to the top of the home page, but it also lets creators of high quality content get paid directly by other users for their work.

While it is clear how zapping incentivizes content creators, it is not as clear what incentives users have to zap. Economists who have studied tipping in non-digital contexts have argued that tipping is not well explained by considerations of future value \citep{azar2020tipping} and have instead of focused on altruistic ``warm glow'' effects \citep{andreoni1990warmglow, crumpler2008warmglow}. However, these are contexts where interactions are generally not repeated and where the payment includes a deterministic price along with an optional tip. On Stacker News, users have repeated interactions with each other and there is no fixed reward for content creators other than tips. The incentive model may therefore be closer to that of a ``pay-what-you-want'' system, in which tipping is rationally used to ensure that producers do not switch to a less advantageous pricing model for consumers \citep{mak2015pay}; in the case of Stacker News, the consequence of not tipping could be a reduction in content production that the user enjoys. Whether users zap due to an altruistic desire to reward content that they enjoy, or by a rational incentive to support the continued production of such content, we do observe a healthy amount of zapping on Stacker News. We also find that the amount of zapping is correlated to objective measures of post quality.

Two additional features of zapping are worth mentioning. First, only 70\% of the zap amount is received by the content creator. 21\% of the zap amount goes to the owner of the territory where the item was posted. Through this, territory owners are additionally incentivized to encourage high quality content in their territories. The remaining 9\% of the zap goes to the daily rewards pool. This 30\% ``tax'' on zaps is a form of sybil resistance: it discourages users from creating sock puppet accounts to zap their own content.

The second feature worth noting is that prior to \gn{NonCustodialDate}, Stacker News acted as a custodial Lightning wallet. Users zapped each other using sats stored in their Stacker News accounts, which they could fund or withdraw at any time using the Lightning network. After \gn{NonCustodialDate}, Stacker News decided to stop holding custody of user funds. Instead, it allowed users to attach their own Lightning wallets to the platform. Users with attached wallets could continue zapping each other with real sats directly over the Lightning Network, but funds would not be stored on Stacker News. Users who did not attach wallets were no longer able to send or receive real sats. Instead, they would send and receive ``Cowboy Credits'', tokens usable only on Stacker News whose value is pegged 1:1 to the value of a sat. Cowboy Credits (CCs) can be used on Stacker News to pay for anything a sat could pay for, including territory billing costs, posting fees, and zaps. In most of our analysis, it is appropriate to treat CCs and sats as interchangeable for user behavior on Stacker News. Where there are differences, we will highlight them in the subsequent analysis.

\paragraph{Daily rewards.} Third, Stacker News incentivizes user behavior with a daily rewards system. As already mentioned, 30\% of posting fees and 9\% of zaps goes to the daily rewards pool. There are three other main sources of daily rewards. First, users can donate directly to the rewards pool. This is not common, and is utilized primarily by users associated with Stacker News, like its founder and employees. Second, users can post anonymously under a dedicated \texttt{anon} account. Any zaps earned by the \texttt{anon} account are sent to the rewards pool. Third, users can pay to boost their own posts. Boosting increases the post's visibility based on the amount paid. When a user pays to boost a post, 70\% of the amount goes to the owner of the territory that the post is in and 30\% goes to the daily rewards pool.

Users are allocated sats from the daily rewards pool according a formula that depends on user behavior in that day. The exact algorithm is unknown to users, but users know that they are rewarded for their highly zapped posts and comments, and for zapping good content early (that is, they are rewarded more if they zap content that other users zap, before the other users zap them). The reward system amplifies the already present incentives to post good content, but it also incentivizes users to help identify and surface good content by zapping good content early.

\paragraph{Web of trust.} Fourth, Stacker News maintains a user reputation system that it calls the ``Web of Trust'' (WoT). The WoT is a modified PageRank algorithm in which trust flows from one user to another when the first user zaps the second, or when the first user zaps an item that the second user already zapped. Trust therefore flows in the direction of preference similarity between users. If we let $\mathbf{t}$ be the $N \times 1$ vector of user trust scores, and $\mathbf{W}$ be the $N \times N$ matrix of user-trust-pairs ($w_{ij}$ is how much user $i$ trusts user $j$), then the trust vector $\mathbf{t}$ solves:
\begin{align}
\mathbf{t} = (1-\alpha) \mathbf{W}^T \mathbf{t} + \alpha \mathbf{t}_0
\end{align}
where $\mathbf{t}_0$ is a seed vector of initial trust scores and $\alpha \in (0,1)$ is called the seed weight. The trust scores can be thought of as the steady state of a diffusion process in which trust particles travel between users at rates governed by $\mathbf{W}$, and in which a fraction $\alpha$ of the existing particles die each period and respawn according to the distribution given in $\mathbf{t}_0$. Trust is calculated separately for each territory, and for active territories the seed vector is zeros for all users except the territory owner, thus giving territory owners a strong influence in other users' trust scores for that territory. There is also a global trust score in which the seed vector is zeros for everyone except Stacker News founders and employees.

Trust matters on Stacker News for a number of reasons. First, the default sort order on the home page ranks posts according to trust-weighted zaps. Thus, the zaps of users with high trust scores have a greater influence on which posts surface to the top of the front page. Trust also influences the amount of rewards a user receives from the daily rewards pool.

Since trust is a metric that is calculated using the entire history of users' interactions with each other, it is difficult for a user to unilaterally improve their trust score in a short period of time. The only way for a user to improve their trust score is through a prolonged period of good behavior, which includes posting good content and zapping good content early. The trust system therefore incentivizes user behavior over longer horizons than posting fees, zaps, and daily rewards.

\paragraph{Market-based content moderation.} Lastly, Stacker News does not employ any content moderators, nor are there formal moderators in any territory other than the territory owners. Instead, Stacker News utilizes a form of market-based content moderation called ``downzaps''. A downzap is like a zap, except the zapper pays to \emph{downweight} the zapped item in any search or sort rankings. If enough users downzap an item, it becomes ``outlawed'' and will be hidden from view for most users.\footnote{Users can opt in to seeing outlawed content in their settings.}  

Getting downzapped or outlawed is not a good outcome for the item creator. It lowers the item's visibility and makes it less likely that the post will accumulate zaps. Since posting requires fees, this makes it more likely that the poster will lose money on the post. Moreover, getting downzapped lowers a user's trust score, which reduces their ability to earn rewards or influence other users' post rankings. 

The requirement for downzaps to be paid with sats is an important feature of this system. It prevents users from frivolously downzapping other users and it prevents users from spinning up sybil accounts to do the same. By requiring a monetary payment for downvoting, Stacker News ensures that only content which is truly objectionable to the user gets downzapped. 
