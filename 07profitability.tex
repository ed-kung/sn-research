\section{Profitability and user retention}

We now test our last hypothesis: that consistently unprofitable users are more likely to exit the platform, whereas profitable users are more likely to stay.

To measure profitability, we only consider whether the user is able to earn enough zaps and rewards to cover their posting costs. We do not consider their own zaps to other users as part of their costs, since zapping is entirely optional. Moreover, some users may have altruistic motives to zap, as discussed in the introduction, so greater zapping could indicate a greater commitment to the platform.

We define a user as inactive if they go four or more consecutive weeks without making any posts, any comments, or zapping any items. The first week of inactivity is considered the week they went inactive. Inactive users can become active again, though we treat this event as exogenous. 

The goal of this section is to estimate how user profitability affects the probability of becoming inactive. We estimate linear probability models of the following form:
\begin{align}
\text{Inactive}_{it} = \alpha \text{Unprofitable}_{it} + \beta \text{Unprofitable}_{it} \times \text{Growth}_{t} + \gamma_{t} + \epsilon_{it}
\end{align}
where $i$ indexes a user and $t$ indexes a week. $\text{Inactive}_{it}$ is a binary indicator for whether user $i$ became inactive on week $t$, $\text{Unprofitable}_{it}$ is a binary indicator for whether the user was unprofitable in the last 8 weeks, and $\text{Growth}_{t}$ is the percentage growth in Bitcoin-USD price in the last 8 weeks. $\gamma_{t}$ is a week fixed effect to capture global time trends, and $\epsilon_{it}$ is the error term. We hypothesize that $\alpha>0$, indicating that users who have been unprofitable are more likely to become inactive, and that $\beta>0$, that this effect is amplified when Bitcoin's price has recently appreciated.

Table \ref{tab_profitability_analysis} reports the results for this regression. Column 1 shows the result when only profitability in the last 8 weeks is included as a regressor. Column 2 additionally controls for the number of items the user posted in the last 8 weeks. Unsurprisingly, users that have posted more items in the last 8 weeks are less likely to become inactive, as it shows a greater commitment to the platform. Including the number of items in the last 8 weeks significantly moderates the effect of unprofitability, suggesting that there are some unprofitable users who post a lot anyway. Column 3 adds week fixed effects to control for global time trends, but the coefficient of interest is not much changed. Lastly, column 4 interacts unprofitability with the Bitcoin price appreciation over the last 8 weeks. 

Using column 4 as the preferred specification, the results show that users who were unprofitable in the last 8 weeks are \gn{UnprofitableExitEffect} percentage points more likely to become inactive than users who were profitable in the last 8 weeks. This is in comparison to a baseline inactive probability of \gn{InactiveRateBaseline}\% per week. In addition, column 4 shows that unprofitable are even more likely to become inactive when the Bitcoin price has recently appreciated. If the Bitcoin price has appreciated by 10 percent in the last 8 weeks, this increases the effect of unprofitability to a \gn{UnprofitableExitBTCGrowthEffect} increased probability of exit. This suggests that users are sensitive to not just their profitability as measured in sats, but also their profitability as measured in purchasing power. 


