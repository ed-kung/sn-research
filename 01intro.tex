\section{Introduction} \label{sec_intro}

Public discourse for matters both mundane and substantive is increasingly mediated through online discussion platforms. \citet{newman2025digitalnews} estimates that usage of social media as a source of news in the U.S. rose from 27\% to 54\% between the years 2013 and 2025, and \citet{mcclain2024pew} shows that the most popular social media platform for political news is a discussion platform, X. There is growing recognition that the incentive structure governing online discussion platforms is a crucial factor determining whether these conversations proceed in healthy or unhealthy directions \citep{aridor2024}. Concerns about online discourse have become prevalent enough that governments around the world are exploring regulatory frameworks for moderating online content, with the EU's Digital Services Act being a primary example.

Most online discussion platforms are free to use. Some may charge for premium features, but these features are usually related to removing ads, content management tools, and features for gaining visibility or monetization. Few platforms charge a direct cost to post, and fewer still allow users to pay each other \emph{directly} for content that they appreciate. Instead, the exchange of value between content creator and content consumer tends to be mediated through indirect mechanisms, such as ad revenue, visibility, and reputation---often referred to as ``clout''---which may indirectly benefit the content creator's endeavors outside the discussion platform proper. The indirect nature of these mechanisms may result in misaligned incentives between content creators and content consumers. For example, content creators may be more incentivized to create content that readers will \emph{click}, rather than create content that readers will actually read, digest, and thank the creator for producing.

There are a few reasons why online discussion platforms may not want to charge for posting or allow users to pay each other. For one, economies of scale and network effects suggest that platforms would want to onboard as many users with as little cost to the user as possible. But another, perhaps less appreciated reason is frictions in the payments infrastructure. The value of any individual post or comment is likely quite small, too small to cover the fixed fees for credit transactions. Fee-less payments networks like Venmo do exist, but they usually have strict volume limits and charge fees to frequent users. Moreover, most popular discussion platforms, like X and Reddit, are international in scope. Routing payments between international users introduces multiple layers of frictions related to currency exchange and international banking regulations. Because of these frictions, no mainstream discussion platform currently charges for posts or allows users to pay each other directly via the platform.

Despite this, there is good reason to believe that charging for posts and allowing users to pay each other directly could increase the quality of content posted. Signaling theory and empirical evidence suggest that posting costs can be an effective anti-spam deterrent, because only posters of high quality content would be willing to pay a cost \citep{joseph2008email, tchernichovski2019pnas}. Although not directly related to discussion platforms, empirical investigations into other types of social media have shown that monetary incentives influence the quantity and quality of content \citep{chen2010google, sun2013blogs, burtch2017ms, elkomboz2023youtube, kerkhof2024youtube}, but the relationships are complex and depend on the form of monetary and social incentives.

To our knowledge, there is little work to date about the impact of monetary incentives in online discussion platforms, despite its relative importance for the quality of social, political, and other kinds of discourse online. This paper aims to fill that gap using a unique dataset from an online discussion platform called Stacker News. Stacker News is an internet message board in the style of Reddit that uses Bitcoin micropayments sent over the Lightning Network. Users can send each other Bitcoin micropayments, called ``zaps'' (usually on the order of 1-100 satoshis, but sometimes ranging into the thousands) to reward each other for posting content that they enjoy and to encourage the production of similar content. The platform also uses micropayments as a form of sybil, spam, and bot resistance. Users must pay a fee to post, which discourages them from posting low quality content or from creating sock puppet accounts. Users who consistently post high quality content can earn enough zaps to turn a profit net of posting fees. The micropayments on Stacker News are made possible by the low cost, instantaneous transaction speed, and international connectivity of the Lightning Network. 

Because payments sent on Stacker News are real money (Bitcoin), it is an ideal setting for studying how financial micro-incentives affect internet discourse. There is variation in posting fees across time, and across the platform's subforums (called territories), allowing us to study how content quality and content quantity changes when posting costs change. Using difference-in-differences regressions, we find that when a territory doubles its posting cost, it reduces the expected weekly number of posts by \gn{PostsCostDoublingEffect}\%, but increases the quality of posts (as measured by zaps received in the first 48 hours) by  