\section{Pay-to-post and content quality}\label{sec_pay_to_post}

We first consider hypothesis 1: whether higher posting costs lead to higher quality posts. To test this, we first need a measure of quality. We consider three measure of post quality: 1) the number of sats the post earns on zaps in the first 48 hours of posting; 2) the number of replies the post generates in the first 48 hours; and 3) a content-based measure of post quality which we describe further below. We focus on the first 48 hours so as not to advantage older posts in the quality measure, and also because \gn{Sats48} percent of all sats earned on zaps are earned in the first 48 hours, and \gn{Comments48} percent of all comments are made within the first 48 hours.

In addition, we apply the following sample selection criteria. We do not include posts in the \texttt{AMA} territory or the \texttt{jobs} territory because these  territories seem to operate on different incentive structures than the other ones.\footnote{\texttt{AMA} is a territory where influential people can make ``Ask Me Anything'' posts. Because these posters are influential, posts in the \texttt{AMA} territory earn an unusually large amount of sats, and the posters are unlikely to be motivated by posting costs. \texttt{jobs} is a territory run by Stacker News in which companies can make job advertisements.} We also do not include a number of special post types that do not follow normal posting rules.\footnote{These include profile posts, which users can post and edit for free, ``saloon'' posts, which are daily discussion threads posted by Stacker News, and freebies---a limited number of reduced-visibility posts that users are allowed to make for free.} Lastly, we do not include any posts made in territories by the owner of the territory. Since 70\% of posting costs go to the territory owner, the incentives for territory owners to post in their own territories differ from that of non-owners. 


