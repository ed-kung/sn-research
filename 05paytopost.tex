\section{Pay-to-post and content quality}\label{sec_pay_to_post}

In this section, we consider our first hypothesis: whether higher posting costs lead to fewer posts in a territory, but higher quality posts. To test this, we consider two measures of post quality: 1) the number of sats the post earns on zaps in the first 48 hours of posting; and 2) the number of replies the post generates in the first 48 hours. We focus on the first 48 hours so as not to advantage older posts in the quality measure, and also because \gn{Sats48} percent of all sats and \gn{Comments48} percent of all comments are earned within the first 48 hours.

In addition, we apply the following sample selection criteria. We do not include posts in the \texttt{AMA} territory or the \texttt{jobs} territory because these  territories seem to operate on different incentive structures than the other ones.\footnote{\texttt{AMA} is a territory where influential people can make ``Ask Me Anything'' posts. Because these posters are influential, posts in the \texttt{AMA} territory earn an unusually large amount of sats, and the posters are unlikely to be motivated by posting costs. \texttt{jobs} is a territory run by Stacker News in which companies can make job advertisements.} We also do not include a number of special post types that do not follow normal posting rules.\footnote{These include profile posts, which users can post and edit for free, ``saloon'' posts, which are daily discussion threads posted by Stacker News, and freebies---a limited number of reduced-visibility posts that users are allowed to make for free.} Lastly, we do not include any posts made in territories by the owner of the territory. Since 70\% of posting costs go to the territory owner, the incentives for territory owners to post in their own territories differ from that of non-owners. 

Let $i$ index a post, let $j$ index the territory that the post was made in, and let $t$ index the week that the post was made. The basic regression that we run is as follows:
\begin{align}
\ln \text{Quality}_{ijt} = \alpha \ln \text{Cost}_{ijt} + \delta_{j} + \gamma_{t} + \epsilon_{ijt} \label{eq_qual_reg}
\end{align}
where $\text{Quality}_{ijt}$ is the measure of post quality, $\text{Cost}_{ijt}$ is the cost (in sats) of the post, $\delta_{j}$ is a set of dummy variables for each territory, and $\gamma_{t}$ is a set of dummy variables for each week. The coefficient $\alpha$ identifies the elasticity between posting cost and post quality.

The regression is a difference-in-differences regression, in which the causal effect of posting cost is identified from differential changes in post quality across territories when territory posting costs change differentially. For example, if the posting cost in the \texttt{econ} territory increases while the posting cost in \texttt{bitcoin} stays the same, and subsequently the average quality of posts increases in \texttt{econ} relative to \texttt{bitcoin}, this would suggest that the change in posting cost likely caused the change in post quality. The coefficient $\alpha$ is identified off the many such changes present in the data.

To give the reader a sense of the variation in posting costs, we plot the posting cost histories for four representative territories in Figure \ref{fig_posting_fee_histories}. The figure shows that there is indeed substantial variation in posting fees both across and within territories. Altogether, the data contains \gn{NumPostingFeeChanges} changes in territory posting fees across the \gn{NumTerritories} territories. In the data we use for the regression, the average posting cost is \gn{AvgPostingCost} sats, the average amount of sats earned in the first 48 hours is \gn{AvgSats48} sats, and the average number of comments in the first 48 hours is \gn{AvgComments48}. There is substantial variation in all three: the standard deviation of the posting cost is \gn{SdPostingCost} sats, of sats earned in the first 48 hours is \gn{SdSats48}, and of comments in the first 48 hours is \gn{SdComments48}. 

Table \ref{tab_sats48_cost_reg} shows results for regressions of specification \eqref{eq_qual_reg} where post quality is measured as zaps in the first 48 hours. Each column of Table \ref{tab_sats48_cost_reg} includes a different set of fixed effects in the regression. Column 1 includes no fixed effects, and is therefore a straight OLS regression of log zaps in the first 48 hours on log posting cost. Column 2 includes territory fixed effects to control for any heterogeneous effects between territories. Column 3 additionally includes week fixed effects to control for global time trends in zapping behavior on Stacker News. Lastly, column 4 includes fixed effects for the identity of the territory owner to control for the zapping behavior of territory owners. For example, some territory owners zap every post that comes into their territory while others do not. Territory-owner fixed effects control for this behavior.\footnote{Note that territory fixed effects do not fully absorb territory owner fixed effects, since a single user can own multiple territories, and territories can change owners over time.}

Table \ref{tab_sats48_cost_reg} shows that there is a statistically significant and robust positive relationship between territory posting fees and post quality as measured by zaps in the first 48 hours. In terms of the magnitude of the effect, Table \ref{tab_sats48_cost_reg} column 4 (our preferred specification) suggests that if a territory doubles its posting cost, then the quality of posts, as measured by zaps in the first 48 hours, is expected to increase by \gn{Zaps48CostDoublingEffect} percent. The relationship is not explained by territory specific effects, such as a spurious correlation between posting fees and the popularity of a territory's topic, nor is it explained by global time trends in posting fees and zap behavior on Stacker News. Nor is it explained by effects related to territory owners---for example, if territory owners who set high fees also tend to zap posts in their territory more generously. The best explanation for the correlation between posting cost and post quality is that users will only choose to post in a higher cost territory if they think that the post is of high enough quality to warrant the cost. One Stacker News user is quoted saying: 

\begin{quote}
``If I put a lot of effort into a post, I know that the posting fee will be offset by the zaps. And even if it doesn't, I'm just happy to show the work I've done and I'm ready to pay for it. On the other hand, when I share a simple link with a few quick comments and quotes, it takes me a few minutes, and I look for the cheaper territory to post it. If I don't find any cheap territory, I'd rather not post, as I don't think paying to post for links is worth the same as paying to post for actual proof-of-work.''\footnote{In this statement, the user is referring to link posts, which are posts that include a link to an external URL, often without any text in the post body. Link posts usually receive less zaps and are considered to be of lower quality because the content is not produced by the user themselves, and are thus less rewarded by zaps.}
\end{quote}

It makes sense that users will only be willing to pay a high posting cost if they think the quality of the post justifies it---either because they will recuperate the cost through zaps or because of a stronger intrinsic desire to post high quality work. However, one may wonder why users don't just choose a low cost territory to post in. There is no rule on Stacker News that says you must post content to the most relevant territory. Moreover, most territories on Stacker News are lightly moderated, so posting content to an irrelevant territory is not likely to result in post removal. However, in practice we observe that users tend to post in territories for which the content is most relevant. This may be for a few reasons. First, users can subscribe to or mute territories.\footnote{Subscribing to a territory means you will receive notifications when a new post is made to that territory. Muting a territory means that you will no longer see posts from that territory.} Thus, posting in a more relevant territory will increase the visibility of your post to users who are most likely to be interested in it. Second, territory owners sometimes add value to their territories by collecting, summarizing, and rewarding top posts. Your post is less likely to be highlighted if it is not relevant to the territory. Lastly, users may have an intrinsic, aesthetic desire to post in the most relevant territory, especially if they view their content to be of high quality.

Next, we present results when the number of comments in the first 48 hours is used to measure quality instead of the number of zaps. The results are presented in Table \ref{tab_comments48_cost_reg}. As with zaps, Table \ref{tab_comments48_cost_reg} demonstrates that there is a robust and statistically significant positive relationship between territory posting cost and the number of comments a post receives in the first 48 hours. Table \ref{tab_comments48_cost_reg}  column 4 implies that when territory posting cost doubles, the number of replies a post in that territory is expected to receive in the first 48 hours increases by \gn{Comments48CostDoublingEffect} percent. 

Now we consider whether territory posting cost also influences the number of posts a territory receives. If users are sensitive to posting cost, then we expect that territory posting fee will be negatively correlated with number of posts. To test this, we run regressions of the following form:
\begin{align}
\ln \text{Posts}_{jt} = \alpha \ln \text{Cost}_{jt} + \delta_{j} + \gamma_{t} + \varepsilon_{jt} \label{eq_quant_reg}
\end{align}
where $j$ indexes a territory and $t$ indexes a week. $\text{Posts}_{jt}$ is the number of posts posted in territory $j$ in week $t$, $\text{Cost}_{jt}$ is territory $j$'s posting fee in week $t$ (measured as the average across days), $\delta_j$ is a set of territory fixed effects, and $\gamma_{t}$ is a set of week fixed effects. The results of this regression, for various combinations of fixed effects, is presented in Table \ref{tab_posts_cost_reg}. As expected, there is a robust and statistically significant negative relationship between territory posting cost and the quantity of posts in a territory. Using column 4 as our preferred specification, the results imply that when the posting cost in a territory doubles, the territory can expect to receive \gn{PostsCostDoublingEffect} percent fewer posts in a week. 

Taken together, the results are both surprising and unsurprising. They are unsurprising in the sense that they conform with economic theory: demand curves slope downwards (when posting costs go up, number of posts goes down), and signaling theory works (when posting costs go up, higher quality posts are made). They are surprising in the sense that even such small micro-incentives (the average posting cost is just \gn{AvgPostingCost} sats, or about \gn{AvgPostingCostCents} cents) are enough to influence user behavior in such a way that post quality is improved. The results suggest that pay-to-post may be an effective mechanism for mediating content quality, even at very small monetary amounts.






