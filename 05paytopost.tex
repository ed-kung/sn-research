\section{Pay-to-post and content quality}\label{sec_pay_to_post}

We first consider hypothesis 1: whether higher posting costs lead to higher quality posts. To test this, we first need a measure of quality. We consider three measure of post quality: 1) the number of sats the post earns on zaps in the first 48 hours of posting; 2) the number of replies the post generates in the first 48 hours; and 3) a content-based measure of post quality which we describe further below. We focus on the first 48 hours so as not to advantage older posts in the quality measure, and also because \gn{Sats48} percent of all sats and \gn{Comments48} percent of all comments are earned within the first 48 hours.

In addition, we apply the following sample selection criteria. We do not include posts in the \texttt{AMA} territory or the \texttt{jobs} territory because these  territories seem to operate on different incentive structures than the other ones.\footnote{\texttt{AMA} is a territory where influential people can make ``Ask Me Anything'' posts. Because these posters are influential, posts in the \texttt{AMA} territory earn an unusually large amount of sats, and the posters are unlikely to be motivated by posting costs. \texttt{jobs} is a territory run by Stacker News in which companies can make job advertisements.} We also do not include a number of special post types that do not follow normal posting rules.\footnote{These include profile posts, which users can post and edit for free, ``saloon'' posts, which are daily discussion threads posted by Stacker News, and freebies---a limited number of reduced-visibility posts that users are allowed to make for free.} Lastly, we do not include any posts made in territories by the owner of the territory. Since 70\% of posting costs go to the territory owner, the incentives for territory owners to post in their own territories differ from that of non-owners. 

Let $i$ index a post, let $j$ index the territory that the post was made in, and let $t$ index the week that the post was made. The basic regression that we run is as follows:
\begin{align}
\ln y_{ijt} = \beta \ln c_{ijt} + \delta_{j} + \gamma_{t} + \epsilon_{ijt}
\end{align}
where $y_{ijt}$ is the measure of post quality (either sats or comments in the first 48 hours), $c_{ijt}$ is the cost (in sats) of the post, $\delta_{j}$ is a set of dummy variables for each territory, and $\gamma_{t}$ is a set of dummy variables for each week. The coefficient $\beta$ identifies the elasticity between posting cost and post quality.

The regression is a difference-in-differences regression, in which the causal effect of posting cost is identified from changes in post quality across territories when territory posting costs change differentially. For example, if the posting cost in the \texttt{econ} territory increases while the posting cost in \texttt{bitcoin} stays the same, and simultaneously the average quality of posts increases in \texttt{econ} relative to \texttt{bitcoin}, this would suggest that the change in posting cost may have had something to do with the change in quality. The coefficient $\beta$ is identified off the many such changes present in the data.

To give the reader a sense of the variation in posting costs, we plot the posting cost histories for four representative territories in Figure \ref{fig_posting_fee_histories}. The figure shows that there is indeed substantial variation in posting fees both across and within territories. Altogether, the data contains \gn{NumPostingFeeChanges} changes in territory posting fees across the \gn{NumTerritories} territories. In addition to variation driven by changes to territory posting costs, there is also variation in posting costs driven by various cost modifiers. For example, if a user wants to post multiple times in the same territory within a short time span, they have to pay a $10\times$ cost modifier for each additional post they make within a 10 minute span. Moreover, there is a cost to uploading large images to the post. In the data we use for the regression, the average posting cost is \gn{AvgPostingCost} sats, the average amount of sats earned in the first 48 hours is \gn{AvgSats48} sats, and the average number of comments in the first 48 hours is \gn{AvgComments48}. There is substantial variation in all three: the standard deviation of the posting cost is \gn{SdPostingCost} sats, of the sats earned in the first 48 hours is \gn{SdSats48}, and of the comments in the first 48 hours is \gn{SdComments48}. 




